\documentclass[a4paper,10pt]{article}
\usepackage{amsmath}
\usepackage{amssymb}
\usepackage[T1]{fontenc}

%%%%%%%%%%%%%%%%%%%%%%%%%%%%%%%%%%%%%%%%%%%%%%%%%%%%
% Author: Thomas Braun
% begin: Sun 9 april 2006
% last edit: --
% License: GPLv2 or later
%%%%%%%%%%%%%%%%%%%%%%%%%%%%%%%%%%%%%%%%%%%%%%%%%%%%%%

\iffalse
\newcommand{\begin{neededpkgs}}[2][]{}
\newcommand{\pkgs}[2][]{}
\fi

%%%%%%%%%%%%%%%%%%%%%%%%%%%%%%%%%%%%%%%%%%%%%%%%%%%%
% Author: Thomas Braun
% begin: Wed 5 april 2006
% last edit: --
% License: GPLv2 or later
%%%%%%%%%%%%%%%%%%%%%%%%%%%%%%%%%%%%%%%%%%%%%%%%%%%%%%

\setlength{\parindent}{0cm} % only to look better if viewed on one page
\pagestyle{empty}

% \newcommand{\page}{\newpage} % default for generating pngs
\newcommand{\page}{} % use this to get it on one page

\newcommand{\mathcommand}[2][]{\ensuremath{#2}\page} % easy mathmode :)
\newcommand{\command}[2][]{#2\page}  % without mathmode

% The optional argument is the desired command which kile should insert
% This can therefore differ from the command used to generate the png,
% Only use it if you really know what you do !!
% if emtpy, the string in the mandatory argument is used as command to insert into kile.

\newenvironment{neededpkgs}[2][]{}{}
\newcommand{\pkgs}[2][]{}
% needed additional packages for this command, see testfile.tex for how to use it


\begin{document}
% 
\mathcommand{\amalg}
\mathcommand{\cup}
\mathcommand{\oplus}
\mathcommand{\times}
\mathcommand{\ast}
\mathcommand{\dagger}
\mathcommand{\oslash}
\mathcommand{\triangleleft}
\mathcommand{\bigcirc}
\mathcommand{\ddagger}
\mathcommand{\otimes}
\mathcommand{\triangleright}
\mathcommand{\bigtriangledown}
\mathcommand{\diamond}
\mathcommand{\pm}
\pkgs{amssymb}\mathcommand{\unlhd}
\mathcommand{\bigtriangleup}
\mathcommand{\div}
\pkgs{amssymb}\mathcommand{\rhd}
\mathcommand{\bullet}
\pkgs{amssymb}\mathcommand{\lhd}
\mathcommand{\setminus}
\mathcommand{\uplus}
\mathcommand{\cap}
\mathcommand{\mp}
\mathcommand{\sqcap}
\mathcommand{\vee}
\mathcommand{\cdot}
\mathcommand{\odot}
\mathcommand{\sqcup}
\mathcommand{\wedge}
\mathcommand{\circ}
\mathcommand{\ominus}
\mathcommand{\star}
\mathcommand{\wr}
\par
\begin{neededpkgs}{amssymb}
\mathcommand{\barwedge}
\mathcommand{\circledcirc}
\mathcommand{\intercal}
\mathcommand{\boxdot}
\mathcommand{\circleddash}
\mathcommand{\leftthreetimes}
\mathcommand{\boxminus}
\mathcommand{\Cup}
\mathcommand{\ltimes}
\mathcommand{\boxplus}
\mathcommand{\curlyvee}
\mathcommand{\rightthreetimes}
\mathcommand{\boxtimes}
\mathcommand{\curlywedge}
\mathcommand{\rtimes}
\mathcommand{\Cap}
\mathcommand{\divideontimes}
\mathcommand{\smallsetminus}
\mathcommand{\centerdot}
\mathcommand{\dotplus}
\mathcommand{\veebar}
\mathcommand{\circledast}
\mathcommand{\doublebarwedge}
\end{neededpkgs}
\par
\mathcommand{\bigcap}
\mathcommand{\bigotimes}
\mathcommand{\bigwedge}
\mathcommand{\prod}
\mathcommand{\bigcup}
\mathcommand{\bigsqcup}
\mathcommand{\coprod}
\mathcommand{\sum}
\mathcommand{\bigodot}
\mathcommand{\biguplus}
\mathcommand{\int}
\mathcommand{\bigoplus}
\mathcommand{\bigvee}
\mathcommand{\oint}
\par
\begin{neededpkgs}{amsmath}
\mathcommand{\iint}
\mathcommand{\iiint}
\mathcommand{\iiiint}
\mathcommand{\idotsint}
\end{neededpkgs}

\end{document}





