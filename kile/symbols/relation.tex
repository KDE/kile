\documentclass[a4paper,10pt]{article}
\usepackage{amssymb}

%%%%%%%%%%%%%%%%%%%%%%%%%%%%%%%%%%%%%%%%%%%%%%%%%%%%
% Author: Thomas Braun
% begin: Sat 8 april 2006
% last edit: --
% License: GPLv2 or later
%%%%%%%%%%%%%%%%%%%%%%%%%%%%%%%%%%%%%%%%%%%%%%%%%%%%%%

%%%%%%%%%%%%%%%%%%%%%%%%%%%%%%%%%%%%%%%%%%%%%%%%%%%%
% Author: Thomas Braun
% begin: Wed 5 april 2006
% last edit: --
% License: GPLv2 or later
%%%%%%%%%%%%%%%%%%%%%%%%%%%%%%%%%%%%%%%%%%%%%%%%%%%%%%

\setlength{\parindent}{0cm} % only to look better if viewed on one page
\pagestyle{empty}

\newcommand{\page}{\newpage} % default for generating pngs
% \newcommand{\page}{} % use this to get it on one page

\newcommand{\mathcommand}[2][]{\ensuremath{#2}\page} % easy mathmode :)
\newcommand{\command}[2][]{#2\page}  % without mathmode

% The optional argument is the desired command which kile should insert
% This can therefore differ from the command used to generate the png,
% Only use it if you really know what you do !!
% if emtpy the string in the mandatory argument is used as command to insert into kile.

\newenvironment{neededpkgs}[2][]{}{}
\newcommand{\pkgs}[2][]{}
% needed additional packages for this command, see testfile.tex for how to use it


\begin{document}

%% begin binary relations
\mathcommand{\approx}
\mathcommand{\equiv}
\mathcommand{\perp}
\mathcommand{\smile}
\mathcommand{\asymp}
\mathcommand{\frown}
\mathcommand{\prec}
\mathcommand{\succ}
\mathcommand{\bowtie}
\pkgs{amssymb} \mathcommand{\Join}
\mathcommand{\preceq}
\mathcommand{\succeq}
\mathcommand{\cong}
\mathcommand{\mid}
\mathcommand{\propto}
\mathcommand{\vdash}
\mathcommand{\dashv}
\mathcommand{\models}
\mathcommand{\sim}
\mathcommand{\doteq}
\mathcommand{\parallel}
\mathcommand{\simeq}
\par
\begin{neededpkgs}{amssymb}
\mathcommand{\approxeq}
\mathcommand{\eqcirc}
\mathcommand{\succapprox}
\mathcommand{\backepsilon}
\mathcommand{\fallingdotseq}
\mathcommand{\succcurlyeq}
\mathcommand{\backsim}
\mathcommand{\multimap}
\mathcommand{\succsim}
\mathcommand{\backsimeq}
\mathcommand{\pitchfork}
\mathcommand{\therefore}
\mathcommand{\because}
\mathcommand{\precapprox}
\mathcommand{\thickapprox}
\mathcommand{\between}
\mathcommand{\preccurlyeq}
\mathcommand{\thicksim}
\mathcommand{\Bumpeq}
\mathcommand{\precsim}
\mathcommand{\varpropto}
\mathcommand{\bumpeq}
\mathcommand{\risingdotseq}
\mathcommand{\Vdash}
\mathcommand{\circeq}
\mathcommand{\shortmid}
\mathcommand{\vDash}
\mathcommand{\curlyeqprec}
\mathcommand{\shortparallel}
\mathcommand{\Vvdash}
\mathcommand{\curlyeqsucc}
\mathcommand{\smallfrown}
\mathcommand{\doteqdot}
\mathcommand{\smallsmile}
\end{neededpkgs}
\par
\begin{neededpkgs}{amssymb}
\mathcommand{\ncong}
\mathcommand{\nshortparallel}
\mathcommand{\nVDash}
\mathcommand{\nmid}
\mathcommand{\nsim}
\mathcommand{\precnapprox}
\mathcommand{\nparallel}
\mathcommand{\nsucc}
\mathcommand{\precnsim}
\mathcommand{\nprec}
\mathcommand{\nsucceq}
\mathcommand{\succnapprox}
\mathcommand{\npreceq}
\mathcommand{\nvDash}
\mathcommand{\succnsim}
\mathcommand{\nshortmid}
\mathcommand{\nvdash}
\end{neededpkgs}
%% end binary relations
% 
%% begin subset and superset relations
\par
\pkgs{amssymb}\mathcommand{\sqsubset}
\mathcommand{\sqsupseteq}
\mathcommand{\supset}
\mathcommand{\sqsubseteq}
\mathcommand{\subset}
\mathcommand{\supseteq}
\pkgs{amssymb}\mathcommand{\sqsupset}
\mathcommand{\subseteq}
\par

\begin{neededpkgs}{amssymb}
\mathcommand{\nsubseteq}
\mathcommand{\subseteqq}
\mathcommand{\supsetneqq}
\mathcommand{\nsupseteq}
\mathcommand{\subsetneq}
\mathcommand{\varsubsetneq}
\mathcommand{\nsupseteqq}
\mathcommand{\subsetneqq}
\mathcommand{\varsubsetneqq}
\mathcommand{\sqsubset}
\mathcommand{\Supset}
\mathcommand{\varsupsetneq}
\mathcommand{\sqsupset}
\mathcommand{\supseteqq}
\mathcommand{\varsupsetneqq}
\mathcommand{\Subset}
\mathcommand{\supsetneq}
\end{neededpkgs}
%  end set relations
\par
% begin inequalities
\mathcommand{\geq}
\mathcommand{\gg}
\mathcommand{\leq}
\mathcommand{\ll}
\mathcommand{\neq}
\par
\begin{neededpkgs}{amssymb}
\mathcommand{\eqslantgtr}
\mathcommand{\gtrdot}
\mathcommand{\lesseqgtr}
\mathcommand{\ngeq}
\mathcommand{\eqslantless}
\mathcommand{\gtreqless}
\mathcommand{\lesseqqgtr}
\mathcommand{\ngeqq}
\mathcommand{\geqq}
\mathcommand{\gtreqqless}
\mathcommand{\lessgtr}
\mathcommand{\ngeqslant}
\mathcommand{\geqslant}
\mathcommand{\gtrless}
\mathcommand{\lesssim}
\mathcommand{\ngtr}
\mathcommand{\ggg}
\mathcommand{\gtrsim}
\mathcommand{\lll}
\mathcommand{\nleq}
\mathcommand{\gnapprox}
\mathcommand{\gvertneqq}
\mathcommand{\lnapprox}
\mathcommand{\nleqq}
\mathcommand{\gneq}
\mathcommand{\leqq}
\mathcommand{\lneq}
\mathcommand{\nleqslant}
\mathcommand{\gneqq}
\mathcommand{\leqslant}
\mathcommand{\lneqq}
\mathcommand{\nless}
\mathcommand{\gnsim}
\mathcommand{\lessapprox}
\mathcommand{\lnsim}
\mathcommand{\gtrapprox}
\mathcommand{\lessdot}
\mathcommand{\lvertneqq}
\end{neededpkgs}
% end inequalities

% begin triangle relations
\begin{neededpkgs}{amssymb}
\mathcommand{\blacktriangleleft}
\mathcommand{\ntrianglelefteq}
\mathcommand{\trianglelefteq}
\mathcommand{\vartriangleleft}
\mathcommand{\blacktriangleright}
\mathcommand{\ntriangleright}
\mathcommand{\triangleq}
\mathcommand{\vartriangleright}
\mathcommand{\ntriangleleft}
\mathcommand{\ntrianglerighteq}
\mathcommand{\trianglerighteq}
\end{neededpkgs}
% end triangle relations

\end{document}
