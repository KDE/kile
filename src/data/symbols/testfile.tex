\documentclass[a4paper,10pt]{article}
\usepackage[T1]{fontenc}

%%%%%%%%%%%%%%%%%%%%%%%%%%%%%%%%%%%%%%%%%%%%%%%%%%%%
% Author: Thomas Braun
% begin: Fri april 7 2006
% last edit: Sun august 13 2006
% License: GPL2 or later
% 
% desc: testfile for the parser
%%%%%%%%%%%%%%%%%%%%%%%%%%%%%%%%%%%%%%%%%%%%%%%%%%%

%%%%%%%%%%%%%%%%%%%%%%%%%%%%%%%%%%%%%%%%%%%%%%%%%%%%
% Author: Thomas Braun
% begin: Wed 5 april 2006
% last edit: --
% License: GPLv2 or later
%%%%%%%%%%%%%%%%%%%%%%%%%%%%%%%%%%%%%%%%%%%%%%%%%%%%%%

\setlength{\parindent}{0cm} % only to look better if viewed on one page
\pagestyle{empty}

\newcommand{\page}{\newpage} % default for generating pngs
% \newcommand{\page}{} % use this to get it on one page

\newcommand{\mathcommand}[2][]{\ensuremath{#2}\page} % easy mathmode :)
\newcommand{\command}[2][]{#2\page}  % without mathmode

% The optional argument is the desired command which kile should insert
% This can therefore differ from the command used to generate the png,
% Only use it if you really know what you do !!
% if emtpy the string in the mandatory argument is used as command to insert into kile.

\newenvironment{neededpkgs}[2][]{}{}
\newcommand{\pkgs}[2][]{}
% needed additional packages for this command, see testfile.tex for how to use it


\begin{document}
\pagestyle{empty}

% - The basic command is \command{\symbolname},  additionaly the command \mathcommand{...} is defined which insert its argument in math mode.
% - Needed packages can be specified using \pkgs[package-opt-argument]{package-name} for a single command or with \begin{neededpkgs}[pkgs-args,]{pkgs,} ... \end{neededpkgs} for all enclosed commands.
% - The \pkgs command has to be in front of the \command command and overrides any pkg specification by the neededpkgs enviroment.
% - The optional argument of \command and \mathcommand can hold a different command for insertion, e.g. useful for stuff like \mathcommand[\stackrel{}{}]{\stackrel{abc}{=}}
%  After writing your custom tex file mysymbols.tex , you have to issue `gesymb mysymbols.tex user` and if everything went well copy the files to "$HOME/.kde/share/apps/kile/mathsymbols/user".
% After restarting kile you should see your icons in the "User defined`` symbol list.
\begin{neededpkgs}{envpkg-blubb}
\command{a}
\pkgs{cmdpkg-blubb}\command{b}
\command{c}
\pkgs[cmdpkg-blubb3-optarg]{cmdpkg-blubb3}\command{d}
\command{dd}
\end{neededpkgs}

\begin{neededpkgs}[envpkg-blubb2-optarg]{envpkg-blubb2}
\command{e}
\pkgs{cmdpkg-blubb2}\command{f}
\command{g}
\pkgs[cmdpkg-blubb5-optarg]{cmdpkg-blubb5}\command{h}
\command{hh}
\end{neededpkgs}

\command{i}
\pkgs{cmdpkg-blubb4}\command{j}
\command{k}
\pkgs[cmdpkg-blubb6-optarg]{cmdpkg-blubb6}\command{l}
\command{m}
\command{n}
\end{document}

% \begin{neededpkgs}[russian,koi8-r,T2C,]{babel,inputenc,fontenc,mathtext}
%   	
%   \end{neededpkgs}
% this would need to include the packages
% \usepackage{mathtext}
% \usepackage[T2C]{fontenc}
% \usepackage[russian]{babel}
% \usepackage[koi8-r]{inputenc}
%  just to explain the format

